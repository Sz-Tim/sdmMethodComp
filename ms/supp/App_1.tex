\documentclass[]{article}
\usepackage{lmodern}
\usepackage{amssymb,amsmath}
\usepackage{ifxetex,ifluatex}
\usepackage{fixltx2e} % provides \textsubscript
\ifnum 0\ifxetex 1\fi\ifluatex 1\fi=0 % if pdftex
  \usepackage[T1]{fontenc}
  \usepackage[utf8]{inputenc}
\else % if luatex or xelatex
  \ifxetex
    \usepackage{mathspec}
  \else
    \usepackage{fontspec}
  \fi
  \defaultfontfeatures{Ligatures=TeX,Scale=MatchLowercase}
\fi
% use upquote if available, for straight quotes in verbatim environments
\IfFileExists{upquote.sty}{\usepackage{upquote}}{}
% use microtype if available
\IfFileExists{microtype.sty}{%
\usepackage{microtype}
\UseMicrotypeSet[protrusion]{basicmath} % disable protrusion for tt fonts
}{}
\usepackage[margin=1in]{geometry}
\usepackage{hyperref}
\hypersetup{unicode=true,
            pdftitle={Appendix 1: Supplemental Methods},
            pdfauthor={Tim M. Szewczyk, Marek Petrik, Jenica M. Allen},
            pdfborder={0 0 0},
            breaklinks=true}
\urlstyle{same}  % don't use monospace font for urls
\usepackage{graphicx,grffile}
\makeatletter
\def\maxwidth{\ifdim\Gin@nat@width>\linewidth\linewidth\else\Gin@nat@width\fi}
\def\maxheight{\ifdim\Gin@nat@height>\textheight\textheight\else\Gin@nat@height\fi}
\makeatother
% Scale images if necessary, so that they will not overflow the page
% margins by default, and it is still possible to overwrite the defaults
% using explicit options in \includegraphics[width, height, ...]{}
\setkeys{Gin}{width=\maxwidth,height=\maxheight,keepaspectratio}
\IfFileExists{parskip.sty}{%
\usepackage{parskip}
}{% else
\setlength{\parindent}{0pt}
\setlength{\parskip}{6pt plus 2pt minus 1pt}
}
\setlength{\emergencystretch}{3em}  % prevent overfull lines
\providecommand{\tightlist}{%
  \setlength{\itemsep}{0pt}\setlength{\parskip}{0pt}}
\setcounter{secnumdepth}{5}
% Redefines (sub)paragraphs to behave more like sections
\ifx\paragraph\undefined\else
\let\oldparagraph\paragraph
\renewcommand{\paragraph}[1]{\oldparagraph{#1}\mbox{}}
\fi
\ifx\subparagraph\undefined\else
\let\oldsubparagraph\subparagraph
\renewcommand{\subparagraph}[1]{\oldsubparagraph{#1}\mbox{}}
\fi

%%% Use protect on footnotes to avoid problems with footnotes in titles
\let\rmarkdownfootnote\footnote%
\def\footnote{\protect\rmarkdownfootnote}

%%% Change title format to be more compact
\usepackage{titling}

% Create subtitle command for use in maketitle
\providecommand{\subtitle}[1]{
  \posttitle{
    \begin{center}\large#1\end{center}
    }
}

\setlength{\droptitle}{-2em}

  \title{Appendix 1: Supplemental Methods}
    \pretitle{\vspace{\droptitle}\centering\huge}
  \posttitle{\par}
  \subtitle{The performance of presence-based and process-based species distribution
models}
  \author{Tim M. Szewczyk, Marek Petrik, Jenica M. Allen}
    \preauthor{\centering\large\emph}
  \postauthor{\par}
    \date{}
    \predate{}\postdate{}
  
\newcommand{\beginsupplement}{\setcounter{table}{0}  \renewcommand{\thetable}{A.\arabic{table}} \setcounter{figure}{0} \renewcommand{\thefigure}{A.\arabic{figure}}}  \usepackage{longtable}  \usepackage{caption}

\begin{document}
\maketitle

{
\setcounter{tocdepth}{1}
\tableofcontents
}
\setcounter{table}{0}  \renewcommand{\thetable}{A.\arabic{table}} \setcounter{figure}{0} \renewcommand{\thefigure}{A.\arabic{figure}}

\begin{center}\rule{0.5\linewidth}{\linethickness}\end{center}

This appendix contains supplemental methods pertaining to the virtual
species generation, species distribution model specifics, and scenario
particulars.

\begin{center}\rule{0.5\linewidth}{\linethickness}\end{center}

\section{General model structure}
\subsection{Integral Projection Model overview}

To generate fully-known true distributions for the virtual species, we
used the general structure of an Integral Projection Model (IPM) to
calculate the intrinsic growth rate \(\lambda\) in each cell of the
gridded landscape, and adapted the regression-based structure of an IPM
into an individual-level, simulation-based cellular automata (CA) model
to produce spatiotemporally dynamic abundance distributions. The
equations below apply to both models.

IPMs use the size distribution, \(z\), of individuals at time \(t\),
along with a kernel \(K(z^{\prime}, z)\), to predict the size
distribution, \(z^{\prime}\), at time \(t+1\): \begin{equation}
n_{t+1}(z^{\prime}) = \int_{\Omega} K(z^{\prime}, z) n_t(z) dz
\end{equation}

Here, \(\Omega\) represents the range of possible sizes for the species
or population. The kernel \(K(z^{\prime}, z)\) is composed of a growth
and survival component, \(P(z, z^{\prime})\), representing the fate of
individuals from time \(t\) to \(t+1\), and a fecundity component,
\(F(z, z^{\prime})\), representing new individuals added between time
\(t\) and \(t+1\). In practice, the integral is approximated using a
discretized transition matrix, and the intrinsic growth rate \(\lambda\)
is calculated as the first eigenvalue of the transition matrix.

The \(P\) and \(F\) kernels are decomposed further into more mechanistic
conditional probabilities and parameters, many of which are functions of
the size distribution \(z\). For example: \begin{equation}
\begin{split}
K(z^{\prime}, z) & = P(z^{\prime}, z) + F(z^{\prime}, z) \\
    & = s(z) g(z^{\prime}|z) + p_{flower}(z) f_{seeds}(z) p_{estab} f_{rcrSize}(z^{\prime})
\end{split}
\end{equation}

where \(s(z)\) is the survival probability of individuals based on size,
\(g(z^{\prime}|z)\) is the probability density of size \(z^{\prime}\)
for an individual of size \(z\), \(p_{flower}(z)\) is the probability
that an individual of size \(z\) produces flowers, \(f_{seeds}(z)\) is
the expected number of seeds produced by an individual of size \(z\)
given that they flower, \(p_{estab}\) is the probability that a seed
germinates and establishes as a new recruit, and
\(f_{rcrSize}(z^{\prime})\) is the expected size distribution of new
recruits at time \(t+1\). On a gridded landscape, the population in each
cell could be modelled independently using the above structure, with
environmental effects incorporated by allowing the environment to
influence parameter values (e.g., \(f_{seeds}\)).

\subsection{Adding a seed bank}

This basic IPM structure is very flexible, allowing for discrete life
stages, reproduction-dependent mortality, and environmental covariates.
To incorporate a seed bank where seeds that do not germinate between
time \(t\) and \(t+1\) may survive to \(t+2\) or beyond, the fecundity
kernel is altered, with seed bank \(B\), such that: \begin{align}
n_{t+1}(z^{\prime}) &= B_t s_{rcrB} p_{estab} f_{rcrSize}(z^{\prime}) + \int_{\Omega}[P(z^{\prime}, z)+F(z^{\prime}, z)] n_t(z) dz \\
F(z^{\prime}, z) &= p_{flower}(z)  f_{seeds}(z) s_{rcrDirect} p_{estab} f_{rcrSize}(z^{\prime}) \\
B_{t+1} &= B_t s_{survB} (1-s_{rcrB})  + p_{flower}(z)  f_{seeds}(z) (1-s_{rcrDirect}) s_{survB} n_t(z) dz
\end{align}

where \(B_t\) is the number of seeds in the seed bank at time \(t\),
\(s_{rcrB}\) is the probability a seed recruits from the seed bank,
\(s_{survB}\) is the probability a seed survives in the seed bank from
time \(t\) to \(t+1\), and \(s_{rcrDirect}\) is the probability a seed
produced in year \(t\) germinates between year \(t\) and \(t+1\). Two
notes about the structure and definitions: 1) a seed added to the seed
bank must fail to recruit in time \(t\) and must also survive from \(t\)
to \(t+1\), and 2) the probability of recruiting is best interpreted as
the probability of germinating and is therefore separate from the
probability of establishing. Both of these could be defined differently
to combine each set of processes, though keeping them separate allows
for seeds to perish.

\subsection{Adding dispersal}

The above equations assume isolated populations in each cell. However,
for a typical plant species, dispersal occurs when seeds move from the
cell where they are produced to a different cell. In an IPM with a seed
bank, this will affect the fecundity kernel \(F(z^{\prime}, z)\) and the
seed bank \(B\). Specifically, the number of seeds in the seed bank in
cell \(i\) at time \(t+1\) will be the number of seeds surviving in the
seed bank \(B_i\) from time \(t\) to \(t+1\), plus the number of seeds
produced in cell \(i\) at time \(t\) that remain in cell \(i\) and do
not recruit directly, plus the number of seeds entering cell \(i\) from
cells \(j = 1, ..., J\) as immigrants in time \(t\) that then fail to
recruit directly: \begin{equation}
    \begin{aligned}
        B_{i,t+1} = \enspace
        & B_{i,t} s_{survSB} (1-s_{rcrSB}) \enspace+  \\
        & \int_{\Omega} p_{flower,i}(z)  f_{seeds,i}(z) (1-p_{emig}) (1-s_{rcrDirect}) s_{survB} n_{i,t}(z) dz \enspace+  \\
        & \sum_{j=1}^{J} \int_{\Omega} [p_{flower,j}(z)  f_{seeds,j}(z) p_{emig} n_{t,j}(z) dz] p_{SDD,ji} (1-s_{rcrDirect}) s_{survB}\\
    \end{aligned}
\end{equation} \begin{equation}
\begin{aligned}
n_{i,t+1}(z^{\prime}) = \enspace
& B_{i,t} s_{rcrSB} p_{estab_i} f_{rcrSize}(z^{\prime}) \enspace+   \\
& \int_{\Omega} [s_i(z) g_i(z^{\prime}|z) + p_{flower_i}(z)  f_{seeds_i}(z) (1-p_{emig}) s_{rcrDirect} p_{estab_i} f_{rcrSize}(z^{\prime})] n_{i,t}(z) dz \enspace +  \\
& \sum_{j=1}^{J} \int_{\Omega} [p_{flower,j}(z)  f_{seeds,j}(z) p_{emig} n_{t, j}(z) dz] p_{SDD,ji} s_{rcrDirect} p_{estab,i} f_{rcrSize}(z^{\prime})\\
\end{aligned}
\end{equation} for each cell \(i\) which is a target cell of each cell
\(j\) of \(J\) cells, where the integral describes the seed production
in each cell \(j\) and \(p_{SDD,ji}\) is the probability that a seed
dispersed from \(j\) lands in \(i\). Note that, as above, seeds added to
the seed bank must survive overwinter as well as fail to recruit
directly.

\begin{center}\rule{0.5\linewidth}{\linethickness}\end{center}

\newpage
\section{Regression equations for the virtual species}

We used the above IPM structure, including a seed bank and dispersal, as
the basis for the virtual species. Thus, the population in each cell
\(i\) has size distribution \(z\), where \(z^{\prime}\) is the size
distribution the next year, and \(\boldsymbol{z_i}\) is a matrix with
columns for: 1, \(z\), \(z^2\), and \(z^3\).

\subsection{Survival}

Annual survival, \(\boldsymbol{s_i}\), was modelled for each individual
in cell \(i\) as a binary outcome (0: mortality; 1: survival) following
a Bernoulli distribution with probability \(\boldsymbol{\psi_{si}}\)
such that: \begin{align}
\boldsymbol{s_i} & \sim Bern(\boldsymbol{\psi_{si}}) \\
logit(\boldsymbol{\psi_{si}}) & = \boldsymbol{z_i}\beta_{s} + \boldsymbol{X_i}\theta_{s}
\end{align} where \(\beta_{s}\) is a vector of covariates for size,
\(\boldsymbol{X_i}\) is a set of cell-level environmental covariates,
and \(\theta_{s}\) is a vector of responses to the environmental
covariates.

\subsection{Growth}

The size distribution, \(\boldsymbol{z^{\prime}_i}\) of individuals in
cell \(i\) for time \(t+1\) was distributed normally about the vector of
expected sizes, \(\boldsymbol{\mu_{gi}}\) with standard deviation
\(\sigma_g\), such that \begin{align}
\boldsymbol{z^{\prime}_i} &\sim Norm(\boldsymbol{\mu_{gi}}, \sigma_g) \\
\boldsymbol{\mu_{gi}} &= \boldsymbol{z_i}\beta_{g} + \boldsymbol{X_i}\theta_{g}
\end{align} where \(\beta_{g}\) is a vector of covariates for size,
\(\boldsymbol{X_i}\) is a set of cell-level environmental covariates,
and \(\theta_{g}\) is a vector of responses to the environmental
covariates.

\subsection{Flowering}

Individual flowering, \(\boldsymbol{l_i}\), was modelled for each
individual in cell \(i\) as a binary outcome (0: no flowers; 1: flowers)
following a Bernoulli distribution with probability
\(\boldsymbol{\psi_{li}}\) such that: \begin{align}
\boldsymbol{l_i} &\sim Bern(\boldsymbol{\psi_{li}}) \\
logit(\boldsymbol{\psi_{li}}) &= \boldsymbol{z_i}\beta_{l} + \boldsymbol{X_i}\theta_{l}
\end{align} where \(\beta_{l}\) is a vector of covariates for size,
\(\boldsymbol{X_i}\) is a set of cell-level environmental covariates,
and \(\theta_{l}\) is a vector of responses to the environmental
covariates.

\subsection{Seeds}

The number of seeds produced, \(\boldsymbol{d_i}\) by each flowering
individual in cell \(i\) for time \(t\) was Poisson distributed about
the vector of expected seed counts, \(\boldsymbol{\mu_{di}}\), such that
\begin{align}
\boldsymbol{d_i} &\sim Poisson(\boldsymbol{\mu_{di}}) \\
log(\boldsymbol{\mu_{di}}) &= \boldsymbol{z_i}\beta_{d} + \boldsymbol{X_i}\theta_{d}
\end{align} where \(\beta_{d}\) is a vector of covariates for size,
\(\boldsymbol{X_i}\) is a set of cell-level environmental covariates,
and \(\theta_{d}\) is a vector of responses to the environmental
covariates. In each cell \(i\), the seeds produced stay in the cell,
emigrate through short distance dispersal, or perish. Seeds that survive
may either enter the seed bank or recruit directly.

\subsection{Dispersal}

The total number of immigrant seeds arriving in a cell, \(D_i\) is
calculated as the sum of the seeds from nearby cells that disperse from
cell \(j\) to cell \(i\), such that: \begin{equation}
D_i = \sum\limits_{j=1}^J \sum\limits_{n=1}^{N_j} \boldsymbol{d_j}p_{emig}p_{ji}
\end{equation} where \(J\) is the number of cells dispersing into \(i\),
\(N_j\) is the number of individuals in cell \(j\), \(d_j\) is the
vector of seed numbers produced by individuals in cell \(j\),
\(p_{emig}\) is the probability that seeds produced in cell \(j\)
emigrate, and \(p_{ji}\) is the probability that a seed emigrating from
cell \(j\) is dispersed to cell \(i\).

\subsection{Recruits}

The number of recruits, \(n_{rcr,i}\) in cell \(i\) in time \(t\), may
originate either from the seed bank, from seeds produced in cell \(i\)
in year \(t\), or from immigrant seeds arriving in year \(t\), such
that: \begin{align}
n_{rcr,i} &= B_i p_{rcrB}p_{est} + \sum\limits_{n=1}^{N_i} \boldsymbol{d_i}(1-p_{emig})p_{rcrDirect}p_{est} + D_i p_{rcrDirect}p_{est}\\
\boldsymbol{z^{\prime}_{rcr,i}} &\sim Norm(\mu_{rcr.z}, \sigma_{rcr.z})
\end{align} where \(B_i\) is the number of seeds in the seed bank,
\(p_{rcrB}\) is the probability that a seed germinates from the seed
bank, \(p_{est}\) is the probability that a new seedling establishes,
\(p_rcrDirect\) is the probability that a seed germinates in the year it
was produced, \(z^{\prime}_{rcr,i}\) is the size distribution of new
recruits in year \(t+1\), \(\mu_{rcr.z}\) is the mean recruit size, and
\(\sigma_{rcr.z}\) is the standard deviation of recruit size.

\subsection{Seed Bank}

Finally, the seed bank for year \(t+1\) is calculated as the sum of the
seeds remaining in the seed bank, the seeds produced in cell \(i\) and
entering into the seed bank, and the immigrant seeds entering into the
seed bank, such that: \begin{equation}
B^{\prime}_i = B_i(1-p_{rcrB})s_B + \sum\limits_{n=1}^{N_i} \boldsymbol{d_i}(1-p_{emig})(1-p_{rcrDirect})s_B + D_i(1-p_{rcrDirect})s_B
\end{equation} where \(s_B\) is the probability that a seed survives in
the seed bank to the next year.

\begin{center}\rule{0.5\linewidth}{\linethickness}\end{center}

\newpage
\section{Species Distribution Model details}

This section of the appendix contains additional information regarding
the structure of the species distribution models. The full R code is
available from \url{https://github.com/Sz-Tim/sdmMethodComp}. Include
more information on the CA\textsubscript{p} model and some on the
CA\textsubscript{i} model.

\begin{center}\rule{0.5\linewidth}{\linethickness}\end{center}

\newpage
\section{Scenario details}

This section of the appendix contains additional information regarding
the data and modelling scenarios. The full R code is available from
\url{https://github.com/Sz-Tim/sdmMethodComp}. Include the table showing
the exact implementation of each scenario.


\end{document}
